\section{Related Works}\label{sec:related}

\subsection{IoT Device Security}
With the increasing popularity of IoT device, many existing works focus on the security issues on IoT devices. In recent years, large number of software vulnerabilities on various types of IoT devices have been identified and many attacks against IoT devices have been revealed. Oluwafemi et al.~\cite{oluwafemi2013experimental} demonstrate the possibility of attacks against compact fluorescent lamps by compromising Internet-enabled home automation systems, showing that non-networked devices might be connected to networked devices and hence can be attacked by remote adversaries. Similarly, Notra et al.~\cite{notra2014experimental} evaluate the several household devices in terms of security and privacy and shows that they can be easily compromised by remote attackers due to security flaws. Ronen et al.~\cite{ronen2016extended} study the multiple smart light device and show attackers can even use smart light devices to remotely exhilarate data from a highly secure building with LIFI communication system. The leaked data can be read by attackers from a distance over 100 meters. Ho et al.~\cite{ho2016smart} examine the security of several home smart locks that can be controlled remotely by user or manufacturer. Their work reveals multiple security flaws in design and implementation of smart locks which can be used by attackers to access private information about users and gain unauthorized access. Sivaraman et al.~\cite{sivaraman2016smart} demonstrate a infiltration attack into home network from Internet via a malicious mobile app. Such attack takes advantage of vulnerability in home router to scan for potential vulnerable IoT devices inside home network, collect information and modify the firewall to allow the external entity to directly attack the IoT device. Moreover, Zhang et al.~\cite{zhang2017dolphinattack} reveals the fact that many home assist devices with speech recognition engine are vulnerable to inaudible attack. With malicious hidden voice commands, attackers are able to remotely manipulate many popular home assist device without user‘s notice. Such unauthorized access poses great risk to users leading to privacy leakage or other security issues. 

To mitigate the risk of various attacks against IoT devices, several works have made their effort to perform automated security analysis on IoT devices. 
Costin et al.~\cite{costin2014large} present the first large-scale analysis of firmware images with static analysis techniques. They build a system to collect a large number of firmware from various device vendors, and unpack all these firmware images into millions of files. Their analysis reveals a large number of vulnerabilities and some of them are even shared by many different devices. Although their work found a great number of vulnerabilities, it still suffers from a limitation of accuracy due to their shallow static analysis approach. To improve accuracy, Zaddach et al.~\cite{zaddach2014avatar} propose a hybrid approach by leveraging both emulator and physical device to perform dynamic analysis. In their approach, firmware instructions are executed inside the emulator but all I/O operations are channeled to the physical devices. While this approach enables full emulation of the target device, execution switch between emulator and physical device is expensive and thus is not scalable. Besides, their implementation is specific to certain types of device, so it is not easy to generalize their system to many different devices. Chen et al.~\cite{chen2016towards} present a dynamic security analysis framework for Linux-based IoT devices. First, it collect a large number of firmwares from various vendors. Then, it automatically unpack and configure the firmware to run in an emulator. Finally it perform a large-scale black-box testing with web exploits on the emulator. However, it is impossible to perform comprehensive security analysis with web exploits. Besides, it is a non-trivial task to resolve software dependence and hardware configuration issues in order to prevent kernel panic during the emulation. However, their approach fails to address these issues well, making it unusable with many firmwares.  

Overall, all existing work in the field focus on attack, mitigation or simple security analysis on IoT devices. None of the them  provides capability of comprehensive security analysis with fuzzing techniques in order to find previously unknown bugs on IoT devices. To the best of our knowledge, we are the first to provide practical fuzzing solution to IoT devices. 


\subsection{Fuzzing}

There is a large number of related works in the research field of fuzzing. To improve seed selection for mutation-based fuzzers, AFLFast~\cite{bohme2017coverage} uses Markov Chain to identify ``low-frequency paths'' and focuses most of effort on these paths because it is more likely to trigger bugs when fuzzing on these ``low-frequency paths''. VUzzer~\cite{rawat2017vuzzer} leverages on control-flow graph to select input. To improve coverage, several works implement dynamic symbolic executions, such as DART~\cite{godefroid2005dart}, SAGE~\cite{godefroid2012sage}, KLEE~\cite{cadar2008klee}, Driller~\cite{stephens2016driller}, CUTE~\cite{sen2005cute} and SYMFUZZ~\cite{cha2015program}. Besides, since dynamic taint analysis can find dependencies between the input and the program logic, it is now widely used in many fuzzers, such as BuzzFuzz~\cite{ganesh2009taint}, Taintscope~\cite{wang2010taintscope}, Dowser~\cite{haller2013dowsing} and Angora~\cite{chen2018angora}. In addition, there are also several works that adopt learning techniques to generate input in fuzzing~(e.g. Skyfire\cite{wang2017skyfire}, Learn\&Fuzz~\cite{godefroid2017learn} and GLADE~\cite{bastani2017synthesizing}).  

As mentioned in previous sections, IoT devices usually have very limited resources, lack source code and require addition configuration for fuzzing. None of these existing work addresses all these challenges. However, unlike all these works, our goal is to adapt existing coverage-based fuzzer to work with binary programs on IoT devices. Specifically, we design and implement \sysname, a lightweight and easy deployable fuzzing framework for binary programs on IoT devices.