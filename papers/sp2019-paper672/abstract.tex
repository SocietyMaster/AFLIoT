\begin{abstract}
With the growing prevalence of Internet of Things~(IoT), security and privacy issues on IoT devices have considerably increased. Due to limited resources, security analysis on IoT devices faces great challenges. On one hand, even though there are several existing works that provide automated solutions to identifying vulnerabilities on IoT devices, they are either inaccurate or not scalable. On the other hand, although sophisticated fuzzing technique has been studied for decades and has proved itself to be very effective in solving the aforementioned problems to find software bugs, none of existing fuzzing approaches could work on IoT devices. 

In this paper, we present \sysname, the first lightweight and deployable fuzzing framework for Linux-based IoT devices. Using binary level instrumentation, \sysname executes the instrumented binary program directly on IoT devices during the fuzzing process. We evaluated \sysname on both benchmarks and real world IoT devices. In total, \sysname identified 157 unique crashes in 10 binary programs with only 10\% performance overhead on average, showing that it is both efficient and effective to find software bugs in binary programs on Linux-based IoT devices.

\end{abstract}